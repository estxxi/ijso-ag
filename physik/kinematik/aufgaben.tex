\documentclass[11pt]{article}
\usepackage[utf8]{inputenc}
\usepackage[ngerman]{babel}

\usepackage{amsmath,amsthm,amssymb,amsfonts}

\usepackage{graphicx}
\graphicspath{{abb/}}
\usepackage{float}
\usepackage{tikz}

\usepackage{fancyhdr} % For headers and footers
\usepackage{geometry}
\usepackage{listings}
\usepackage{hyperref}
\hypersetup{
    linkcolor=blue,     
    urlcolor=cyan,
}

\geometry{
    a4paper, % Change this if you intend to print on a different paper size, such as letter paper.
    left=20mm,
    right=20mm,
    top=30mm,
    bottom=30mm,
}

\title{Kinematik - Aufgaben}
\author{Emil Staikov}
\date{}

\begin{document}
\maketitle
\section{Aufgaben aus der IJSO}
\textbf{1.} (aus dem IJSOquiz 2018)
Tina trainiert Tennis mit einer Ballmaschine, die ihr Bälle von der anderen Spielfeldseite zuspielt. Der Tennisball wird dabei von Bodenhöhe schräg nach oben abgeschossen und bewegt sich auf der Bahn einer Parabel. \\
Von welcher der folgenden Größen hängt die Zeit vom Abschuss bis zum Auftreffen in Tinas Spielfeld direkt ab, wenn Luftreibung keine Rolle spielt? \\

(1) vom Betrag der Geschwindigkeit beim Abschuss \\

(2) von der horizontalen Geschwindigkeitskomponente beim Abschuss \\

(3) von der vertikalen Geschwindigkeitskomponente beim Abschuss \\

(4) von der Masse des Tennisballs \\\\
\textbf{2.} 


\pagebreak

\section{Lösungen}
\textbf{1.:} (3) Für eine Parabelbahn des Tennisballs bestimmt allein die vertikale Geschwindigkeitskomponente $v_z$ die
Flugzeit, da mit der Fallbeschleunigung $g$ die Flugzeit als $t = 2v_z /g$ berechnet werden kann.

\end{document}