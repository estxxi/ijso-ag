\documentclass[11pt]{article}
\usepackage[utf8]{inputenc}
\usepackage[ngerman]{babel}

\usepackage{amsmath,amsthm,amssymb,amsfonts}

\usepackage{graphicx}
\usepackage{float}
\usepackage{tikz}

\usepackage{fancyhdr} % For headers and footers
\usepackage{geometry}
\usepackage{listings}
\usepackage{hyperref}
\hypersetup{
    linkcolor=blue,     
    urlcolor=cyan,
}

\geometry{
    a4paper, % Change this if you intend to print on a different paper size, such as letter paper.
    left=20mm,
    right=20mm,
    top=30mm,
    bottom=30mm,
}

\newcount\colveccount
\newcommand*\colvec[1]{
        \global\colveccount#1
        \begin{pmatrix}
        \colvecnext
}
\def\colvecnext#1{
        #1
        \global\advance\colveccount-1
        \ifnum\colveccount>0
                \\
                \expandafter\colvecnext
        \else
                \end{pmatrix}
        \fi
}

\title{Dynamik - Energie (Rotation)}
\author{Emil Staikov}
\date{31. Mai 2021}

\begin{document}
\maketitle
MIt einer Rotationsbewegung können wir wie bei der Translationbewegung eine kinetische Energie assoziieren. Dafür nutzen wir die uns schon bekannte Formel für kinetische Energie, wenden sie aber auf die Tangentialgeschwindigkeit $v_T$ eines rotierenden Massepunkts an: 
\begin{equation*}
        E_{kin} = \frac{1}{2} m v_T^2
\end{equation*}
Zwischen Tangential- und Winkelgeschwindgkeit besteht der bekannte Zusammenhang $v_T = r\omega$, wir setzen also ein 
\begin{equation*}
        \frac{1}{2} m v_T^2 = \frac{1}{2}m (r\omega)^2 = \frac{1}{2}mr^2 \omega^2
\end{equation*}
Der Term $mr^2$ entspricht dem Trägheitsmoment $I$, folglich gilt 
\begin{equation*}
        E_{kin} = \frac{1}{2} I \omega^2
\end{equation*} 
Wir erkennen wieder die Analogien zwischen Größen der Rotation und Translation, wenn wir die Ausdrücke für kinetische Energie von Rotation und Translation vergleichen: 
\begin{equation*}
        E_{kin, t} = \frac{1}{2} m v^2 \quad \quad \quad E_{kin, r} = \frac{1}{2} I \omega^2
\end{equation*}
In der Formel entspricht die Masse dem Trägheitsmoment und die Geschwindigkeit der Winkelgeschwindgkeit. 

\end{document}