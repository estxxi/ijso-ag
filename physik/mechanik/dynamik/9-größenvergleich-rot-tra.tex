\documentclass[11pt]{article}
\usepackage[utf8]{inputenc}
\usepackage[ngerman]{babel}

\usepackage{amsmath,amsthm,amssymb,amsfonts}

\usepackage{graphicx}
\usepackage{float}
\usepackage{tikz}

\usepackage{fancyhdr} % For headers and footers
\usepackage{geometry}
\usepackage{listings}
\usepackage{hyperref}
\hypersetup{
    linkcolor=blue,     
    urlcolor=cyan,
}

\geometry{
    a4paper, % Change this if you intend to print on a different paper size, such as letter paper.
    left=20mm,
    right=20mm,
    top=30mm,
    bottom=30mm,
}

\newcount\colveccount
\newcommand*\colvec[1]{
        \global\colveccount#1
        \begin{pmatrix}
        \colvecnext
}
\def\colvecnext#1{
        #1
        \global\advance\colveccount-1
        \ifnum\colveccount>0
                \\
                \expandafter\colvecnext
        \else
                \end{pmatrix}
        \fi
}

\renewcommand{\arraystretch}{1.5}

\title{Größenvergleich zwischen Rotation und Translation}
\author{Emil Staikov}
\date{}

\begin{document}
\maketitle
In den bisherigen Ausarbeitungen haben wir verschiedene Äquivalenzen zwischen Größen in der Rotations- und Translationsbewegung gefunden, hier halten wir sie tabellarisch fest. \\\\
\begin{center}
    \begin{tabular}{|l|l|}
        \hline
        \textbf{Translation} & \textbf{Rotation} \\
        \hline
        Weg $s$ & Winkel $\varphi$ \\
        \hline
        Geschwindigkeit $v = \frac{\Delta s}{\Delta t}$ & Winkelgeschwindigkeit $\omega = \frac{\Delta \varphi}{\Delta t}$ \\
        \hline 
        Beschleunigung $a = \frac{\Delta v}{\Delta t}$ & Winkelbeschleunigung $\alpha = \frac{\Delta \omega}{\Delta t}$ \\
        \hline  
        Kraft $F$ & Drehmoment $M = rF\sin\theta$ ($\theta = \angle\vec{r}\vec{F}$) \\ 
        \hline 
        Masse $m$ & Trägheitsmoment $I = \sum_{i=0}^n m_ir_i^2$ \\ 
        \hline 
        2. Newton'sches Axiom $F = ma$ & 2. Newton'sches Axiom $M=I\alpha$ \\ 
        \hline 
        Kin En.der Translation $E_{kin} = \frac{1}{2} m v^2$ & Kin. En. der Rotation $E_{kin} = \frac{1}{2}I\omega^2$ \\ 
        \hline 
        Impuls $p = mv$ & Drehimpuls $L = I\omega$ \\
        \hline 
        Impulserhaltung $\sum F = 0 \implies p = const.$ & Drehimpulserhaltung $\sum M = 0 \implies L = const.$\\
        \hline 
    \end{tabular}
\end{center}


\end{document}