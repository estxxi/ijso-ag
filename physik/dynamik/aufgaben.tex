\documentclass[11pt]{article}
\usepackage[utf8]{inputenc}
\usepackage[ngerman]{babel}

\usepackage{amsmath,amsthm,amssymb,amsfonts}

\usepackage{graphicx}
\graphicspath{{abb/}}
\usepackage{float}
\usepackage{tikz}

\usepackage{fancyhdr} % For headers and footers
\usepackage{geometry}
\usepackage{listings}
\usepackage{hyperref}
\hypersetup{
    linkcolor=blue,     
    urlcolor=cyan,
}

\geometry{
    a4paper, % Change this if you intend to print on a different paper size, such as letter paper.
    left=20mm,
    right=20mm,
    top=30mm,
    bottom=30mm,
}

\title{Dynamik - Aufgaben}
\author{Emil Staikov}
\date{}

\begin{document}
\maketitle
\section{Aufgaben aus der IJSO}
\textbf{1.} (aus den Übungsaufgaben zur Vorbereitung auf die Klausurrunde 2019 des IPN) \\
Otto von Guericke ist als Begründer der Pneumatik, der Lehre von den Bewegungen und
Gleichgewichtszuständen der Luft, bekannt. Bei dem so genannten Magdeburger
Halbkugelversuch ließ er Mitte des 17. Jahrhunderts an beiden Seiten einer evakuierten Kugel
(bestehend aus zwei metallischen Halbkugeln, die, einmal leergepumpt, von dem äußeren
Luftdruck zusammengehalten wurden) jeweils acht Pferde einspannen, die versuchen sollten, die
Halbkugeln auseinander zu reißen. Damit wollte er die Kraft des Luftdrucks demonstrieren. Die 16
Pferde schafften es nicht, die Kugeln auseinander zu bringen. Hätte Otto von Guericke dieselbe
Krafteinwirkung auch mit weniger Pferden zeigen können? \\

(1) Nein, da schon 16 Pferde es nicht geschafft haben, die Halbkugeln auseinander zu bringen. \\

(2) Ja, man hätte die Pferde auf der einen Seite durch einen fest verankerten Widerstand ersetzen können. \\

(3) Nein, das wäre nicht effektvoll genug gewesen. \\

(4) Ja, aber es hätten auf jeden Fall auf beiden Seiten die gleiche Anzahl an Pferden eingespannt werden müssen. \\

\pagebreak

\section{Lösungen}
\textbf{1.:} (2) Acht Pferde auf nur einer Seite hätten gereicht, da ein fester Widerstand auf der anderen Seite nach Newtons drittem Axiom dieselbe Gegenkraft ergeben hätte. 

\end{document}